% Options for packages loaded elsewhere
\PassOptionsToPackage{unicode}{hyperref}
\PassOptionsToPackage{hyphens}{url}
%
\documentclass[
]{article}
\usepackage{amsmath,amssymb}
\usepackage{iftex}
\ifPDFTeX
  \usepackage[T1]{fontenc}
  \usepackage[utf8]{inputenc}
  \usepackage{textcomp} % provide euro and other symbols
\else % if luatex or xetex
  \usepackage{unicode-math} % this also loads fontspec
  \defaultfontfeatures{Scale=MatchLowercase}
  \defaultfontfeatures[\rmfamily]{Ligatures=TeX,Scale=1}
\fi
\usepackage{lmodern}
\ifPDFTeX\else
  % xetex/luatex font selection
\fi
% Use upquote if available, for straight quotes in verbatim environments
\IfFileExists{upquote.sty}{\usepackage{upquote}}{}
\IfFileExists{microtype.sty}{% use microtype if available
  \usepackage[]{microtype}
  \UseMicrotypeSet[protrusion]{basicmath} % disable protrusion for tt fonts
}{}
\makeatletter
\@ifundefined{KOMAClassName}{% if non-KOMA class
  \IfFileExists{parskip.sty}{%
    \usepackage{parskip}
  }{% else
    \setlength{\parindent}{0pt}
    \setlength{\parskip}{6pt plus 2pt minus 1pt}}
}{% if KOMA class
  \KOMAoptions{parskip=half}}
\makeatother
\usepackage{xcolor}
\usepackage[margin=1in]{geometry}
\usepackage{color}
\usepackage{fancyvrb}
\newcommand{\VerbBar}{|}
\newcommand{\VERB}{\Verb[commandchars=\\\{\}]}
\DefineVerbatimEnvironment{Highlighting}{Verbatim}{commandchars=\\\{\}}
% Add ',fontsize=\small' for more characters per line
\usepackage{framed}
\definecolor{shadecolor}{RGB}{248,248,248}
\newenvironment{Shaded}{\begin{snugshade}}{\end{snugshade}}
\newcommand{\AlertTok}[1]{\textcolor[rgb]{0.94,0.16,0.16}{#1}}
\newcommand{\AnnotationTok}[1]{\textcolor[rgb]{0.56,0.35,0.01}{\textbf{\textit{#1}}}}
\newcommand{\AttributeTok}[1]{\textcolor[rgb]{0.13,0.29,0.53}{#1}}
\newcommand{\BaseNTok}[1]{\textcolor[rgb]{0.00,0.00,0.81}{#1}}
\newcommand{\BuiltInTok}[1]{#1}
\newcommand{\CharTok}[1]{\textcolor[rgb]{0.31,0.60,0.02}{#1}}
\newcommand{\CommentTok}[1]{\textcolor[rgb]{0.56,0.35,0.01}{\textit{#1}}}
\newcommand{\CommentVarTok}[1]{\textcolor[rgb]{0.56,0.35,0.01}{\textbf{\textit{#1}}}}
\newcommand{\ConstantTok}[1]{\textcolor[rgb]{0.56,0.35,0.01}{#1}}
\newcommand{\ControlFlowTok}[1]{\textcolor[rgb]{0.13,0.29,0.53}{\textbf{#1}}}
\newcommand{\DataTypeTok}[1]{\textcolor[rgb]{0.13,0.29,0.53}{#1}}
\newcommand{\DecValTok}[1]{\textcolor[rgb]{0.00,0.00,0.81}{#1}}
\newcommand{\DocumentationTok}[1]{\textcolor[rgb]{0.56,0.35,0.01}{\textbf{\textit{#1}}}}
\newcommand{\ErrorTok}[1]{\textcolor[rgb]{0.64,0.00,0.00}{\textbf{#1}}}
\newcommand{\ExtensionTok}[1]{#1}
\newcommand{\FloatTok}[1]{\textcolor[rgb]{0.00,0.00,0.81}{#1}}
\newcommand{\FunctionTok}[1]{\textcolor[rgb]{0.13,0.29,0.53}{\textbf{#1}}}
\newcommand{\ImportTok}[1]{#1}
\newcommand{\InformationTok}[1]{\textcolor[rgb]{0.56,0.35,0.01}{\textbf{\textit{#1}}}}
\newcommand{\KeywordTok}[1]{\textcolor[rgb]{0.13,0.29,0.53}{\textbf{#1}}}
\newcommand{\NormalTok}[1]{#1}
\newcommand{\OperatorTok}[1]{\textcolor[rgb]{0.81,0.36,0.00}{\textbf{#1}}}
\newcommand{\OtherTok}[1]{\textcolor[rgb]{0.56,0.35,0.01}{#1}}
\newcommand{\PreprocessorTok}[1]{\textcolor[rgb]{0.56,0.35,0.01}{\textit{#1}}}
\newcommand{\RegionMarkerTok}[1]{#1}
\newcommand{\SpecialCharTok}[1]{\textcolor[rgb]{0.81,0.36,0.00}{\textbf{#1}}}
\newcommand{\SpecialStringTok}[1]{\textcolor[rgb]{0.31,0.60,0.02}{#1}}
\newcommand{\StringTok}[1]{\textcolor[rgb]{0.31,0.60,0.02}{#1}}
\newcommand{\VariableTok}[1]{\textcolor[rgb]{0.00,0.00,0.00}{#1}}
\newcommand{\VerbatimStringTok}[1]{\textcolor[rgb]{0.31,0.60,0.02}{#1}}
\newcommand{\WarningTok}[1]{\textcolor[rgb]{0.56,0.35,0.01}{\textbf{\textit{#1}}}}
\usepackage{graphicx}
\makeatletter
\def\maxwidth{\ifdim\Gin@nat@width>\linewidth\linewidth\else\Gin@nat@width\fi}
\def\maxheight{\ifdim\Gin@nat@height>\textheight\textheight\else\Gin@nat@height\fi}
\makeatother
% Scale images if necessary, so that they will not overflow the page
% margins by default, and it is still possible to overwrite the defaults
% using explicit options in \includegraphics[width, height, ...]{}
\setkeys{Gin}{width=\maxwidth,height=\maxheight,keepaspectratio}
% Set default figure placement to htbp
\makeatletter
\def\fps@figure{htbp}
\makeatother
\setlength{\emergencystretch}{3em} % prevent overfull lines
\providecommand{\tightlist}{%
  \setlength{\itemsep}{0pt}\setlength{\parskip}{0pt}}
\setcounter{secnumdepth}{-\maxdimen} % remove section numbering
\ifLuaTeX
  \usepackage{selnolig}  % disable illegal ligatures
\fi
\IfFileExists{bookmark.sty}{\usepackage{bookmark}}{\usepackage{hyperref}}
\IfFileExists{xurl.sty}{\usepackage{xurl}}{} % add URL line breaks if available
\urlstyle{same}
\hypersetup{
  pdftitle={HW 2 Student},
  pdfauthor={Andy Ackerman},
  hidelinks,
  pdfcreator={LaTeX via pandoc}}

\title{HW 2 Student}
\author{Andy Ackerman}
\date{10/17/2023}

\begin{document}
\maketitle

This homework is meant to illustrate the methods of classification
algorithms as well as their potential pitfalls. In class, we
demonstrated K-Nearest-Neighbors using the \texttt{iris} dataset. Today
I will give you a different subset of this same data, and you will train
a KNN classifier.

\hypertarget{section}{%
\section{}\label{section}}

Above, I have given you a training-testing partition. Train the KNN with
\(K = 5\) on the training data and use this to classify the 50 test
observations. Once you have classified the test observations, create a
contingency table -- like we did in class -- to evaluate which
observations your algorithm is misclassifying.

\begin{Shaded}
\begin{Highlighting}[]
\NormalTok{pr }\OtherTok{\textless{}{-}} \FunctionTok{knn}\NormalTok{(iris\_train,iris\_test,}\AttributeTok{cl=}\NormalTok{iris\_target\_category,}\AttributeTok{k=}\DecValTok{5}\NormalTok{)}
\NormalTok{tab }\OtherTok{\textless{}{-}}\FunctionTok{table}\NormalTok{(pr, iris\_test\_category)}
\NormalTok{tab}
\end{Highlighting}
\end{Shaded}

\begin{verbatim}
##             iris_test_category
## pr           setosa versicolor virginica
##   setosa          5          0         0
##   versicolor      0         25         0
##   virginica       0         11         9
\end{verbatim}

\begin{Shaded}
\begin{Highlighting}[]
\NormalTok{accuracy }\OtherTok{\textless{}{-}} \ControlFlowTok{function}\NormalTok{(x)\{}
  \FunctionTok{sum}\NormalTok{(}\FunctionTok{diag}\NormalTok{(x)}\SpecialCharTok{/}\NormalTok{(}\FunctionTok{sum}\NormalTok{(}\FunctionTok{rowSums}\NormalTok{(x)))) }\SpecialCharTok{*} \DecValTok{100}
\NormalTok{\}}
\FunctionTok{accuracy}\NormalTok{(tab)}
\end{Highlighting}
\end{Shaded}

\begin{verbatim}
## [1] 78
\end{verbatim}

\hypertarget{section-1}{%
\section{}\label{section-1}}

Discuss your results. If you have done this correctly, you should have a
classification error rate that is roughly 20\% higher than what we
observed in class. Why is this the case? In particular run a summary of
the \texttt{iris\_test\_category} as well as
\texttt{iris\_target\_category} and discuss how this plays a role in
your answer.

\emph{STUDENT INPUT}

\begin{Shaded}
\begin{Highlighting}[]
\FunctionTok{summary}\NormalTok{(iris\_test\_category)}
\end{Highlighting}
\end{Shaded}

\begin{verbatim}
##     setosa versicolor  virginica 
##          5         36          9
\end{verbatim}

\begin{Shaded}
\begin{Highlighting}[]
\FunctionTok{summary}\NormalTok{(iris\_target\_category)}
\end{Highlighting}
\end{Shaded}

\begin{verbatim}
##     setosa versicolor  virginica 
##         45         14         41
\end{verbatim}

```

The test and training categories seem to be not sampled randomly.
versicolor is much more present in the test data than the training data.
Because of this there are a lot of versicolor being tested that come
back from the tests virginica; because there are not a representative
amount of virginica in the training and test data.

Choice of \(K\) can also influence this classifier. Why would choosing
\(K = 6\) not be advisable for this data?

\emph{STUDENT INPUT}

\hypertarget{choosing-k-6-would-not-make-much-sense-because-then-you-can-have-an-instance-where-you-essentially-have-a-tie.-this-is-not-good-because-then-r-just-picks-which-side-to-classify-it-as-and-this-is-not-as-good-as-making-it-the-5-nearest-or-7-nearest.}{%
\section{Choosing K = 6 would not make much sense because then you can
have an instance where you essentially have a ``tie''. This is not good
because then R just picks which side to classify it as and this is not
as good as making it the 5 nearest or 7
nearest.}\label{choosing-k-6-would-not-make-much-sense-because-then-you-can-have-an-instance-where-you-essentially-have-a-tie.-this-is-not-good-because-then-r-just-picks-which-side-to-classify-it-as-and-this-is-not-as-good-as-making-it-the-5-nearest-or-7-nearest.}}

Build a github repository to store your homework assignments. Share the
link in this file.

\emph{STUDENT INPUT}

\end{document}
